\documentclass[12pt,oneside,final]{siuethesis}
\usepackage{microtype} % (optional) for more beautiful typesetting
\usepackage{graphicx} 
\usepackage{hyperref} %makes links clickable
\hypersetup{colorlinks,citecolor=black,filecolor=black,linkcolor=blue,urlcolor=black} %good for electronic copy
\hypersetup{colorlinks,citecolor=black,filecolor=black,linkcolor=black,urlcolor=black}%required for paper graduate school copy
%\usepackage[alphabetic]{amsrefs} %required if using amsrefs, comment out if using bibtex


%% controls numbering of theorems
%% this can be configured to your advisor's taste
\newtheorem{theorem}{Theorem}[chapter] %theorem number resets each chapter
\newtheorem{conclusion}[theorem]{Conclusion}
\newtheorem{condition}[theorem]{Condition}
%% conjectures, corollary, defn, etc. numbered sequentially from beginning of chapters
\newtheorem{conjecture}[theorem]{Conjecture} 
\newtheorem{corollary}[theorem]{Corollary}
\newtheorem{example}[theorem]{Example} 
\newtheorem{lemma}[theorem]{Lemma}
\newtheorem{proposition}[theorem]{Proposition}
\newtheorem{solution}[theorem]{Solution}
\theoremstyle{definition}
\newtheorem{definition}[theorem]{Definition}


\author{Ima Student}
\title{On the Principles Which Turn Coffee into Theorems Although This Could Be a Very Long Title and That Would Also Be OK}
\advisor{John Q.\ Faculty} %% or \advisor{Dr.}{John Q.\ Faculty}
\secondreader{Karl Gauss} %% or \secondreader{Dr.}{Karl Gauss}
\thirdreader{Karl Gauss, Jr.}
%\fourthreader{Karl Gauss, Sr.}
%\fifthreader{Karl Gauss, Sr.}
%\secondadvisor{Karl Gauss} %if you haves two advisors (rare) then use this line also and pass the option `twoadvisors' to the class
%\abstracttext{Chairperson: The Honorable Jill Smith} %optional -- you can use this to override the text on the abstract page; the grad school default is built-in
\submitdate{October, 2013} %date the month/year submitted to grad school, use a comma between
\copyrightyear{2013} %optional, but required if copyrighted

%% all of these are optional; defaults are shown
%\major{Mathematics} 
%\degree{Master of Science} %can be used to specify M.A., etc.
%\highestdegree{Bachelor of Science} %used if the author already has another graduate degree
%\department{Mathematics and Statistics} 
%\departmentname{Department}
%\refname{REFERENCES} 


\begin{document}
\maketitle 

\frontmatter %signals single spacing/roman numeral pagination

\copyrightpage %optional

%%% abstracts are optional
\begin{abstract}
Lorem ipsum dolor sit amet, consectetur adipisicing elit, sed do eiusmod tempor incididunt ut labore et dolore magna aliqua. Ut enim ad minim veniam, quis nostrud exercitation ullamco laboris nisi ut aliquip ex ea commodo consequat. Duis aute irure dolor in reprehenderit in voluptate velit esse cillum dolore eu fugiat nulla pariatur. Excepteur sint occaecat cupidatat non proident, sunt in culpa qui officia deserunt mollit anim id est laborum.
\end{abstract}

\begin{acknowledgements}
Lorem ipsum dolor sit amet, consectetur adipisicing elit, sed do eiusmod tempor incididunt ut labore et dolore magna aliqua. Ut enim ad minim veniam, quis nostrud exercitation ullamco laboris nisi ut alidquip ex ea commodo consequat. Duis aute irure dolor in reprehenderit in voluptate velit esse cillum dolore eu fugiat nulla pariatur. Excepteur sint occaecat cupidatat non proident, sunt in culpa qui officia deserunt mollit anim id est laborum.
\end{acknowledgements}

\tableofcontents

\cleardoublepage %cause correct numbering of list of figures

\listoffigures %print list of figures page
\listoftables %print list of figures page

\mainmatter %signals single spacing/arabic numeral paginations


\chapter{INTRODUCTION}  %% chapter titles must be typed in all caps to conform with regulations
\thispagestyle{empty} %omits page number from first page of chapter 1


%% text of the thesis follows

Vestibulum nec risus nunc, bibendum consectetur justo. Proin nec nisl neque, ut porttitor diam. Duis quam tellus, dapibus semper congue eu, vestibulum ac nulla. Suspendisse semper elit a ipsum dignissim eu mollis tellus tristique. Curabitur scelerisque, lectus vel semper fringilla, libero sapien euismod eros, in consectetur erat felis et sem. In hac habitasse platea dictumst. Aenean facilisis enim ut ipsum convallis at gravida quam mattis. Etiam ultrices, est ac posuere elementum, massa orci viverra diam, at vulputate diam risus nec lacus. Etiam eu nisi lacus. Vestibulum ante ipsum primis in faucibus orci luctus et ultrices posuere cubilia Curae; Sed et lobortis arcu. Phasellus orci elit, fringilla et mattis et, consequat quis orci. Curabitur mauris risus, rhoncus vel pharetra ac, tincidunt scelerisque mi. Vestibulum ante ipsum primis in faucibus orci luctus et ultrices posuere cubilia Curae;

Aenean ligula turpis, egestas non pellentesque et, ultricies at leo. Suspendisse tincidunt velit at urna bibendum adipiscing. Nullam ultrices fermentum magna ut tristique. Aliquam sit amet leo urna. Cras vel aliquam tortor. Nullam malesuada sem in quam rhoncus nec posuere dolor dictum. Suspendisse potenti. Suspendisse eget dolor eros, vel ultricies velit. Fusce sit amet risus turpis. Donec placerat dui eu felis convallis dignissim.

Quisque viverra volutpat varius. Integer molestie porttitor risus, eu fermentum massa tristique a. Donec ullamcorper, risus ac posuere ornare, ligula purus ornare metus, eu interdum mauris sapien vitae odio. Lorem ipsum dolor sit amet, consectetur adipiscing elit. Proin dictum mattis sapien, eu accumsan purus consectetur eu. Fusce porta vehicula arcu. Pellentesque lobortis erat mollis risus dictum auctor. Suspendisse potenti. Maecenas at lectus risus. Sed porta enim a metus ultrices ultricies ut vel urna. Lorem ipsum dolor sit amet, consectetur adipiscing elit. Phasellus cursus, arcu quis congue sagittis, nisi justo fermentum orci, condimentum vulputate arcu quam id metus. Donec vulputate ipsum id dolor blandit rutrum. Pellentesque quis massa eu est ultricies dapibus eu egestas quam. Proin metus dui, placerat ut interdum vitae, dignissim eu purus. In eu tellus vitae tellus aliquet tristique. Vivamus elit nisl, feugiat id tincidunt a, tempus ac magna. Donec ut ipsum sit amet eros sollicitudin ultrices. Cras feugiat aliquet massa, vitae fermentum eros varius sit amet.




Suspendisse commodo ante eget lorem vehicula et tempor nisi sodales. Sed sed tortor nec risus faucibus tincidunt. Integer a pellentesque lorem. Fusce vestibulum suscipit libero, vitae venenatis sem iaculis non. Etiam elit sem, porttitor nec lobortis vitae, vestibulum eu diam. Phasellus eleifend lectus at augue pretium vitae facilisis arcu dictum. Vestibulum ac metus lacus, vehicula consequat risus. Nulla commodo magna feugiat libero tempor non accumsan lectus cursus. Morbi blandit aliquam ipsum quis interdum. Praesent a augue quis magna tristique fringilla vel nec nibh. Donec sed purus dignissim sem faucibus elementum. Donec elit elit, tristique eu vestibulum vitae, hendrerit eget nulla. Fusce euismod risus ut justo faucibus eu viverra augue vulputate.

Praesent in neque a mauris dignissim ultricies sed eget felis. Aliquam hendrerit quam vitae metus sagittis fermentum. In viverra lobortis arcu. Phasellus suscipit auctor leo, ac vehicula felis auctor eu. Nunc est nisl, aliquet ac adipiscing vitae, blandit nec dolor. Integer commodo porta sodales. Quisque cursus erat sed risus venenatis dignissim quis eget elit. Phasellus at nibh libero, et pellentesque justo. Quisque a mauris felis, et tincidunt sapien. Vivamus eget metus vitae augue dignissim tristique ut sed velit. Curabitur et tortor ut orci commodo pharetra. Duis sodales porta diam, non blandit nisi bibendum at. Ut molestie mi eget felis fermentum a varius lectus malesuada. Aliquam felis enim, suscipit id aliquet ac, faucibus id velit.

\chapter{PRELIMINARIES A VERY LONG TITLE WHICH SHOULD BREAK INTO TWO LINES IN THE TABLE OF CONTENTS}

%% for very long section titles, you MUST insert slashes at possible break points or the title will not break
\section{Riemann Surfaces and Differential Forms could\ potentially\ be\ a\ very\ long\ heading,\ which\ might\ cause\ problems, and we should check to see that this is properly indented}

In this section, we define what a Riemann surface is and discuss differential forms on Riemann surfaces. To define a Riemann surface, we need a couple of definitions relating to complex manifolds.

\subsection{This is a heading of another level}

Here is some text.  There might be mathematics here such as \[ \int_\Omega d\omega \] or something else.  In this section we'll illustrate some different headings and the like.  Riemann surfaces are fun.

\subsubsection{A lower heading}

And some more text.

\section{Blah blah blah}

asdfadsf


We now have the tools to define a Riemann surface.

\begin{definition}
A complex manifold $X$ is called a Riemann surface if it is a one complex dimensional connected holomorphic manifold. \cite{shuai}
\end{definition}

In our last example, we showed that $\hat{\mathbb{C}}$ is a complex manifold. Since $\hat{\mathbb{C}}$ is connected and has one-complex dimension, $\hat{\mathbb{C}}$ is a Riemann surface. Riemann surfaces will be used as domains of the minimal surfaces discussed in this thesis.  A special type of Riemann surface if formed by the solution set of a two variable polynomial equation. These types of Riemann surface will be useful later on. The following proposition describes these types of Riemann surfaces.

\begin{proposition}
Fix eight unique points $x_i \in \mathbb{C}$. The set
\begin{equation}\label{polynomial}
X=\left\{ \left( x,y\right) \in \hat{
\mathbb{C}}^{2}:y^{2}=\prod\limits_{i=1}^{8}\left( x-x_{i}\right) \right\},
\end{equation}
is a Riemann surface.
\end{proposition}
We must note that the point $(\infty,\infty)$ is a solution to the equation and is a point in $X$.

\begin{proof}
Put 
\begin{equation*}
P\left( x\right) :=\prod\limits_{i=1}^{8}\left( x-x_{i}\right).
\end{equation*}
First, we must show that $X$ is a complex holomorphic manifold. Since $X$ is a subset of a Hausdorff space $\hat{\mathbb{C}}\times \hat{\mathbb{C}}$, we have that $X$ is Hausdorff (using the metric topology). We now need to build the complex manifold structure. For points $\left( x,y_{0}\right) $ such that $y_{0}\neq 0,\infty $ we take the open set $B_{\epsilon }\left( x\right) \times B_{\epsilon
}\left( y_{0}\right)$. We want to find an $\epsilon $ small enough so that the projection $\pi _{1}\left( x,y_{0}\right) =x$ is a homeomorphism. Let $
\epsilon _{1}$ be chosen small enough so that $B_{\epsilon _{1}}\left(y_{0}\right) $ does not contain $-y$ for all $y\in B_{\epsilon _{1}}\left(
y_{0}\right).$ Let $\epsilon _{2}$ be chosen small enough so that $x_{1},x_{2},...,x_{8}$ are not included in $B_{\epsilon _{2}}\left( x\right)
.$ Put $\epsilon =\min \left\{ \epsilon _{1},\epsilon _{2}\right\}$. Then $\pi _{1}$ is bijective in $B_{\epsilon }\left( x\right) \times B_{\epsilon
}\left( y_{0}\right) $. Since $\pi _{1}$ is bicontinuous, $\pi_{1}$ is a homeomorphism. For points $\left( x,0\right) $ we need different
open sets. Since $P\left( x\right) $ has eight unique zeros. we can see that
\begin{equation*}
P^{^{\prime }}\left( x\right) =\sum_{i=1}^{8}\prod\limits_{j\neq i}\left(
x-x_{j}\right)
\end{equation*}
is nonzero at these zeros of $P\left( x\right)$. The Implicit Function Theorem guarantees us small neighborhoods about these points such that the
projection $\pi _{2}\left( x,y\right) =y$ is bijective. For the point $\left( \infty ,\infty \right) ,$ we take the function $\pi _{\infty }\left(
x,y\right) =\frac{1}{x},$ where $\pi _{\infty }\left( \infty ,\infty \right) =0$ as the local coordinate. The open set that we define the local
coordinate in is $$\left\{ \left( x,y\right) :\left\vert x\right\vert>\left\vert x_{i}\right\vert ,i=1,...,8\right\}.$$ This projection is bijective. Now let $U$ be be a set in the cover of the type containing a point $\left( x,y\right) $ such that $y\neq 0$ and $V$ be
the type in the cover where it contains a point $\left( x,0\right).$ If $U\cap V\neq \emptyset$, the mapping 
\begin{equation*}
\pi _{2}\circ \pi _{1}^{-1}\left( x\right) =\sqrt{\prod\limits_{i=1}^{8}
\left( x-x_{i}\right) }
\end{equation*}
is single valued, bijective, and holomorphic. Thus, we have shown that $X$ is a complex manifold. Now we show that $X$ is a Riemann surface. To show that $X$ is connected, we show that $X$ is path connected. Let $\left(x_{1},y_{1}\right) $ and $\left( x_{2},y_{2}\right) $ be two points. We project to the $y$ coordinate. Since$\sqrt{P\left( x\right) }$ is continuous, we can travel along this path until we arrive at $y_{2}$. When
we inverse project, we will either be at $\left( x_{2},y_{2}\right) $ or $\left( -x_{2},y_{2}\right)$ since the square root is a multivalued function. If the later occurs, we can make an analytic continuation. We do this by looping once around a zero. Then, when we inverse project, we arrive at $\left( x_{2},y_{2}\right)$. Hence, $X$ is path connected. Now we show that $X$ has one complex dimension. When developing the complex manifold structure, we showed that the projections $\pi _{1}$ or $\pi _{2}$ are homeomorphisms in small enough balls around any point. Thus $X$ is locally homeomorphic to $\mathbb{C}$. Since $X$ is connected and locally homeomorphic to a one-dimensional complex space, $X$ has one complex dimension. Hence $X$ is a Riemann surface.
\end{proof}


\chapter{CONCLUSION}


Phasellus nulla justo \cite{MR1415998}, aliquet nec lobortis vitae, congue sit amet velit. Nulla facilisis consectetur sapien, non mollis neque ultricies tempor. Cras tempus volutpat mi at venenatis. Pellentesque sit amet mi orci. Pellentesque vulputate pellentesque purus at tincidunt. Aenean sed ante non purus ullamcorper sagittis. Nunc varius, nisi in consequat mollis, lorem odio blandit nisl, tempus sollicitudin ante est vel nunc. Aliquam erat volutpat. Cum sociis natoque penatibus et magnis dis parturient montes, nascetur ridiculus mus. Maecenas non ultrices purus. Morbi elementum auctor gravida.

Aenean gravida faucibus libero at consectetur. Donec commodo quam at turpis feugiat vulputate nec vel ligula. Nulla ut mi leo. Integer consequat condimentum cursus. Duis leo diam, tempor sed dignissim sed, aliquet id quam. Maecenas cursus venenatis nisl, id viverra turpis dapibus non. Maecenas non lorem a libero cursus blandit. Vivamus aliquet vulputate tellus non blandit. Sed egestas laoreet fringilla. Vivamus volutpat nisl id eros lacinia accumsan. Suspendisse imperdiet hendrerit velit, et aliquet tortor gravida sed.



\references %single spacing / arabic numeral paginations, adds "REFERENCES" to table of contents

%%%% for bibtex

%If you want to use bibtex  use the following lines, where your .bib file is called 'yourbib.bib'

\bibliographystyle{alpha}
\bibliography{yourbib}


%%%% for amsrefs

%\begin{biblist}[\BibHangIndent] %graduate school aesthetics :) require a hanging indent
%\bib{shuai}{book}{
%    AUTHOR = {Farkas, H. M. and Kra, I.},
%     TITLE = {Riemann surfaces},
%    SERIES = {Graduate Texts in Mathematics},
%    VOLUME = {71},
%   EDITION = {Second},
% PUBLISHER = {Springer-Verlag},
%   ADDRESS = {New York},
%      YEAR = {1992},
%     PAGES = {xvi+363},
%      ISBN = {0-387-97703-1},
%   MRCLASS = {30Fxx (14-01 14H55)},
%  MRNUMBER = {1139765 (93a:30047)},
%}
%
%\bib{hao}{article}{
%   author={Hao, Xiaoling},
%   author={Sun, Jiong},
%   title={Regular Sturm-Liouville operators with transmission conditions at
%   finite interior discontinuous points},
%   journal={J. Math. Sci. Adv. Appl.},
%   volume={4},
%   date={2010},
%   number={2},
%   pages={265--277},
%   issn={0974-5750},
%   review={\MR{2666452 (2011f:34054)}},
%}
%
%\end{biblist}

% If you have only a single appendix, do it this way.

%\singleappendix
%
%\chapter*{The Only One}
%
%Phasellus nulla justo, aliquet nec lobortis vitae, congue sit amet velit. Nulla facilisis consectetur sapien, non mollis neque ultricies tempor. Cras tempus volutpat mi at venenatis. Pellentesque sit amet mi orci. Pellentesque vulputate pellentesque purus at tincidunt. Aenean sed ante non purus ullamcorper sagittis. Nunc varius, nisi in consequat mollis, lorem odio blandit nisl, tempus sollicitudin ante est vel nunc. Aliquam erat volutpat. Cum sociis natoque penatibus et magnis dis parturient montes, nascetur ridiculus mus. Maecenas non ultrices purus. Morbi elementum auctor gravida.
%
%Aenean gravida faucibus libero at consectetur. Donec commodo quam at turpis feugiat vulputate nec vel ligula. Nulla ut mi leo. Integer consequat condimentum cursus. Duis leo diam, tempor sed dignissim sed, aliquet id quam. Maecenas cursus venenatis nisl, id viverra turpis dapibus non. Maecenas non lorem a libero cursus blandit. Vivamus aliquet vulputate tellus non blandit. Sed egestas laoreet fringilla. Vivamus volutpat nisl id eros lacinia accumsan. Suspendisse imperdiet hendrerit velit, et aliquet tortor gravida sed.

%Otherwise, do your appendices this way.

\multipleappendices

\chapter{First One}

Phasellus nulla justo, aliquet nec lobortis vitae, congue sit amet velit. Nulla facilisis consectetur sapien, non mollis neque ultricies tempor. Cras tempus volutpat mi at venenatis. Pellentesque sit amet mi orci. Pellentesque vulputate pellentesque purus at tincidunt. Aenean sed ante non purus ullamcorper sagittis. Nunc varius, nisi in consequat mollis, lorem odio blandit nisl, tempus sollicitudin ante est vel nunc. Aliquam erat volutpat. Cum sociis natoque penatibus et magnis dis parturient montes, nascetur ridiculus mus. Maecenas non ultrices purus. Morbi elementum auctor gravida.

Aenean gravida faucibus libero at consectetur. Donec commodo quam at turpis feugiat vulputate nec vel ligula. Nulla ut mi leo. Integer consequat condimentum cursus. Duis leo diam, tempor sed dignissim sed, aliquet id quam. Maecenas cursus venenatis nisl, id viverra turpis dapibus non. Maecenas non lorem a libero cursus blandit. Vivamus aliquet vulputate tellus non blandit. Sed egestas laoreet fringilla. Vivamus volutpat nisl id eros lacinia accumsan. Suspendisse imperdiet hendrerit velit, et aliquet tortor gravida sed.


\chapter{Second One}

Phasellus nulla justo, aliquet nec lobortis vitae, congue sit amet velit. Nulla facilisis consectetur sapien, non mollis neque ultricies tempor. Cras tempus volutpat mi at venenatis. Pellentesque sit amet mi orci. Pellentesque vulputate pellentesque purus at tincidunt. Aenean sed ante non purus ullamcorper sagittis. Nunc varius, nisi in consequat mollis, lorem odio blandit nisl, tempus sollicitudin ante est vel nunc. Aliquam erat volutpat. Cum sociis natoque penatibus et magnis dis parturient montes, nascetur ridiculus mus. Maecenas non ultrices purus. Morbi elementum auctor gravida.
d
Aenean gravida faucibus libero at consectetur. Donec commodo quam at turpis feugiat vulputate nec vel ligula. Nulla ut mi leo. Integer consequat condimentum cursus. Duis leo diam, tempor sed dignissim sed, aliquet id quam. Maecenas cursus venenatis nisl, id viverra turpis dapibus non. Maecenas non lorem a libero cursus blandit. Vivamus aliquet vulputate tellus non blandit. Sed egestas laoreet fringilla. Vivamus volutpat nisl id eros lacinia accumsan. Suspendisse imperdiet hendrerit velit, et aliquet tortor gravida sed. 









\end{document}
